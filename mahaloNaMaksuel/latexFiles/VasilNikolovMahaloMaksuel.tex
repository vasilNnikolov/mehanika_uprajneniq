\documentclass[aps, prb, twocolumn, a4paper, floatfix, reprint]{revtex4-2}
\usepackage[%
    margin=10mm,% ако не си принтира 10мм не изглежда грозно, а може да събереш повече текст
    % showframe=true,%
    ]{geometry}
\usepackage[T1,T2A]{fontenc}
\usepackage[utf8]{inputenc}
\usepackage[main=bulgarian, english]{babel}
\usepackage{float}
\AtBeginDocument{\selectlanguage{bulgarian}}
\newcommand{\degree}{^{\circ}}
\usepackage{amsmath}
\usepackage{graphics}
\usepackage{graphicx}
\usepackage{siunitx}
\graphicspath{{.}}
\newcommand{\abs}[1]{\lvert#1\rvert}
\let\phi\varphi
\usepackage{booktabs} % от тук се използва само \midrule може и без него 
%\usepackage{adjustbox} % това може да се използва, за да „смаляваш“ широки таблици
%\usepackage{tabularx} % дефинира колона X в среда tabularx която добавя празно място така че цялата таблица да запълни определена ширина
\usepackage{dcolumn}
\newcolumntype{d}[1]{D{.}{.}{#1}}
\usepackage[unicode=true,pdfusetitle]{hyperref}


\makeatletter
\renewcommand{\Dated@name}{}%
\makeatother



\begin{document}
\title{Махало на Максуел}
\author{Васил Николов}
\noaffiliation
\date{25.12.2021}
\maketitle

\section{Цел на упражнението}
Да се изследва поведението на махалото на Максуел и да се измери инерчният му момент, както и този на пръстените, които могат да се прикачат към него. 

\section{Експериментална установка}
Махалото на Максуел представлява метален цилиндър, през центъра на който преминава тънка метална ос с фиксиран радиус. На оста от двете страни са намотани тънки неразтегливи нишки. Горните краища на нишките са закрепени на една и съща височина, а между тях има електромагнит и фотоклетка, която засича кога махалото е пуснато, и пуска таймер. В долната част на уреда има втора фотоклетка, която засича преминаването на махалото и спира таймера. Тъи като махалото има значим инерчен момент то пада с ускорение $a$, значително по малко от земното ускорение $g$. $a$ зависи от лесно измерими параметри на системата като радиусът на оста на навиване на нишката $R$ и масата на махалото $m$. Ускорението зависи и от инерчният момент на махалото, и когато измерим времето за падане от фиксирана височина може да се намери ускорението и оттам инерчният момент. 

\section{Теоретичен анализ}
Нека инерчният момент на махалото е $I$, радиусът на оста, около която се навиват нишките е $R$ и сумата от двете сили на опън е $T$. Ако системата се движи с ускорение $a$, то
\begin{gather*}
    mg - T = ma \\
    TR = I\frac{a}{R} \\
    \Rightarrow I=mR^2(\frac{g}{a} - 1) \label{eq:1} \tag{1}
\end{gather*}
Тъй като движението е равноускорително 
\begin{equation*}
    a = 2\Delta h/t^2 \label{eq:2} \tag{2}
\end{equation*}
където $\Delta h$ е височината, от която пада махалото, а $t$ е времето, отчетено от установката. 

\section{Експериментални резултати}
Използвайки формули \eqref{eq:1} и \eqref{eq:2} можем да пресметнем инерчните моменти на махалото при окачени различни дискове. Резултатите са представени в Таблица 1.  

\begin{table}[H]
    \label{tab:1}
    \caption{Експериментални данни}
    \begin{ruledtabular}
        \begin{tabular}{*{4}{c}}
            \multicolumn{1}{p{2cm}}{Маса на диск, g} &
            \multicolumn{1}{p{2cm}}{Време за падане, s} &
            \multicolumn{1}{p{2cm}}{Ускорение, $cm.s^{-2}$} &
            \multicolumn{1}{p{2cm}}{Общ инерчен момент} \\[2pt] 
            \midrule
            0 & 1.341 & 46.69 & $\num{9.36E-5}$ \\ 
            255.9 & 2.044 & 19.61 & $\num{5.43e-4}$ \\
            395 & 2.177 & 16.96 & $\num{9.28E-5}$ \\
            514 & 2.205 & 16.318 & $\num{1.04E-3}$ \\
        \end{tabular}
    \end{ruledtabular}
\end{table}
\end{document}
