\documentclass[aps, prb, twocolumn, a4paper, floatfix, reprint]{revtex4-2}
\usepackage[%
    margin=10mm,% ако не си принтира 10мм не изглежда грозно, а може да събереш повече текст
    % showframe=true,%
    ]{geometry}
\usepackage[T1,T2A]{fontenc}
\usepackage[utf8]{inputenc}
\usepackage[main=bulgarian, english]{babel}
\AtBeginDocument{\selectlanguage{bulgarian}}
\newcommand{\degree}{^{\circ}}
\usepackage{amsmath}
\newcommand{\abs}[1]{\lvert#1\rvert}
\let\phi\varphi
\usepackage{booktabs} % от тук се използва само \midrule може и без него 
%\usepackage{adjustbox} % това може да се използва, за да „смаляваш“ широки таблици
%\usepackage{tabularx} % дефинира колона X в среда tabularx която добавя празно място така че цялата таблица да запълни определена ширина
\usepackage{dcolumn}
\newcolumntype{d}[1]{D{.}{.}{#1}}
\usepackage[unicode=true,pdfusetitle]{hyperref}


\makeatletter
\renewcommand{\Dated@name}{}%
\makeatother



\begin{document}
\title{Закон за запазване на импулса. Централен удар на сфери}
\author{Васил Николов}
\noaffiliation
\date{21.11.2021}
\maketitle

\section{Цел на упражнението}
Да се измери коефициентът на възстановяване на система от две метални топчета, окачени на дълги неразтегливи нишки и да се провери законът за запазване на импулса.


\section{Експериментална установка}
Две метални сфери с маси $m_1 = 180.5$~g и $m_2=103.7$~g са окачени на нишки така, че когато отклоним голямата сфера на определен ъгъл ($\phi_0 = 13\degree$) и след това я пуснем, ударът между двете сфери да е централен. След удара трябва да измерим ъглите на максимално отклонение на всяка от сферите. Тъй като малка сфера отново се удря в голямата, за да измерим всички нужни ъгли ще трябва да пуснем голямата сфера два пъти. Първият път гледаме на какъв $\phi_{2}$ се отклонява малката сфера. Втория път хващаме малката сфера преди да се удари повторно в голяма и гледаме на какъв ъгъл $\phi_{1}$ се отклонява голямата сфера.


\section{Теоретична обосновка}
Коефициентът на възстановяване за тази система се дефинира като: \eqref{eq:1}
\begin{gather*}
    K=\frac{\abs{u_2 - u_1}}{\abs{v_2 - v_1}},
\end{gather*}
където $v_1$ и $v_2$ са началните скорости на сферите, а $u_1$ и $u_2$ са крайните им скорости\footnote{Под начална и крайна скорост се има в предвид скоростта непосредствено преди и след удара}. В конкретния случай малката сфера е неподвижна, така че $v_{2} = 0$.

При измерване на максималният ъгъл на отклонение на топчетата след удара, съответно $\phi_1$ и $\phi_2$, от закона за запазване на енергията (ЗЗЕ) следва, че
\begin{gather}
    u_1 = 2\sqrt{lg}\sin\frac{\phi_1}{2} \nonumber\\
    u_2 = 2\sqrt{lg}\sin\frac{\phi_2}{2} \nonumber\\
    K   = \frac{\sin\frac{\phi_2}{2} + \sin\frac{\phi_1}{2}}{\sin\frac{\phi_0}{2}} \label{eq:1}
\end{gather}

За доказване на закона за запазване на импулса трябва да проверим дали
\begin{gather}
    m_1\sin\frac{\phi_0}{2} = m_1\sin\frac{\phi_1}{2} + m_2\sin\frac{\phi_2}{2} \label{eq:2}
\end{gather}

За да проверим това трябва да сметнем стойностите на лявата и дясната страна на равенството, както и експерименталната грешка, която се смята по формулата
\begin{gather*}
    \Delta\left(m\sin\frac{\phi}{2}\right) = \sin\left(\frac{\phi}{2}\right)\Delta m + \frac{1}{2}m\cos\left(\frac{\phi}{2}\right)\Delta \phi
\end{gather*}


\section{Експериментални данни и резултати}
Измерването се повтаря 20 пъти, като първите 10 се измерва единият ъгъл на отклонение, а останалите -- другият ъгъл. Измерените стойности са представени в таблица~\ref{tab:1}.
\begin{table}[ht]
    \caption{\label{tab:1} Измерени стойности за ъглите на отклонение след удара на двете сфери. $\phi_{1}$ и $\phi_{2}$ са ъглите на отклонение съответно на голяма и малката сфера.}
    \begin{ruledtabular}
        \begin{tabular}{rd{1.2}d{2.2}}
            \multicolumn{1}{c}{№}             &
            \multicolumn{1}{c}{$\phi_1$, deg} &
            \multicolumn{1}{c}{$\phi_2$, deg}                \\[2pt]
            \midrule
            1                                 & 4.50 & 12.75 \\
            2                                 & 4.75 & 13.00 \\
            3                                 & 4.50 & 12.25 \\
            4                                 & 4.25 & 12.75 \\
            5                                 & 4.25 & 13.25 \\
            6                                 & 4.25 & 13.00 \\
            7                                 & 4.25 & 13.50 \\
            8                                 & 4.25 & 13.25 \\
            9                                 & 4.25 & 13.50 \\
            10                                & 4.00 & 13.75 \\
        \end{tabular}
    \end{ruledtabular}
\end{table}

От таблицата смятаме средните стойности на ъглите на отклонение:
\begin{gather*}
    \phi_{1} = \bar{\phi}_{1} \pm \Delta\phi = (4.20 \pm 0.25)\degree,\\
    \phi_{2} = \bar{\phi}_{2} \pm \Delta\phi = (13.1 \pm 0.25)\degree.
\end{gather*}

Заместваме в израза за коеф. на възстановяване \eqref{eq:1} и получаваме:
\begin{gather*}
    K = 0.67 \pm 8 \%
\end{gather*}

За проверка на закона за запазване на импулса трябва да се пресметне лявата и дясната част на равенство \eqref{eq:2}, съответно LHS и RHS.
\begin{gather*}
    \text{LHS} = (20.4 \pm 0.8)~\text{g}\\
    \text{RHS} = (18.6 \pm 1.3)~\text{g}
\end{gather*}

В рамките на грешката двете стойности съвпадат, както се и очаква от закона за запазване на импулса.

\section{Резултати}

Измерен е коефициентът на възстановяване на системата $K = 0.67 \pm 8\%$, и експерименталните данни са в съгласие с закона за запазване на импулса.

\end{document}