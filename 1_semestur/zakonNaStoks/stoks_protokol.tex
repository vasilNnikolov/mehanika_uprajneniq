\documentclass[aps, prb, twocolumn, a4paper]{revtex4-2}
\usepackage[T1,T2A]{fontenc}
\usepackage[utf8]{inputenc}
\usepackage[main=bulgarian, english]{babel}

\usepackage{amsmath}
\begin{document}

\title{Закон на Стокс}
\author{Васил Николов}
\noaffiliation
\date{\today}

\maketitle

\section{Цел на експеримента}
Да се измери визкозитетът на глицерин при дадена температура като се измери терминалната скорост на топчета, падащи в него.

\section{Теоретична обосновка}
% схема на експеримента

Нека $m$, $r$ и $v$ са съответно маса, радиус и моментна скорост на топчето. $r$ се измерва с точност 
$F_g$ - сила на тежестта\\
$F_a$ - архимедова сила \\ 
$F_{drag}(v)$ - сила на съпротивление в зависимост от скоростта 

\begin{gather*}
    F_g = mg \\
    F_a = \frac{4}{3}\pi r^3\rho_{liq}g \\
    F_{drag}(v) = 6\pi\eta rv(1 + 2.4\frac{r}{R})\\
    \eta = \frac{(m-\frac{4}{3}\pi r^3\rho_{liq})g}{6\pi rv(1+2.4\frac{r}{R})}
\end{gather*}
Нека частта от тръбата, в която премаме, че топчето се е движило с постоянна скорост, е $L$, а времето, което измерваме, е $t$. 

\begin{gather*}
    v = \frac{L}{t} \\
    \eta = \frac{(m-\frac{4}{3}\pi r^3\rho_{liq})gt}{6\pi r(1+2.4\frac{r}{R})L}
\end{gather*}
За да получим крайния резултат осредняваме всички стойности на $\eta$. 

\section{Експериментална установка}

Във висока вертикална тръба, пълна с глицерин, се пускат оловни топчета с различни размери. Измерва се тяхната терминална скорост като се мери времето, за което изминават дадено фиксирано разстояние. Началото на измерването започва ~20 см под повърхността на глицерина, така че топчето да е достигнало терминалната си скорост.

\section{Екперименталлни данни и резултати}
За конкретната експериментална установка $L=26cm$, а плътността на глицерина е 1200 $kg/m^3$. Радиусът на тръбата е $R=2.6 cm$ 

\begin{table}[ht]
\begin{tabular}{llllll}
    N  & M, mg & D, mm & t, s  & v, m/s & $\eta, Pa*s$\\
1  & 60    & 2.145 & 16.41 & 1.58   & 1.49                          \\
2  & 59    & 2.115 & 15.89 & 1.64   & 1.44                          \\
3  & 44.5  & 1.935 & 18.66 & 1.39   & 1.40                          \\
4  & 57    & 2.155 & 16.12 & 1.55   & 1.37                          \\
5  & 42    & 1.94  & 19.29 & 1.30   & 1.35                          \\
6  & 50    & 2.04  & 16.97 & 1.47   & 1.35                          \\
7  & 40    & 1.93  & 19.61 & 1.27   & 1.31                          \\
8  & 55    & 1.855 & 16.57 & 1.51   & 1.67                          \\
9  & 38    & 1.865 & 20.23 & 1.24   & 1.34                          \\
10 & 66    & 2.1   & 16.03 & 1.56   & 1.66                     
\end{tabular}
\end{table}
Средностатистическата стойност на визкозитета е $\bar{\eta}=1.44 Pa*s\pm 3\% $. Тази стойност е близка до табличните стойности за визкозитета на глицерина $\eta_{table} = 1.417 Pa*s$ при $20^\circ C$. 

\section{Възможни източници на грешка}
Възможни източници на систематична грешка са
\begin{itemize}
    \item Балонче въздух, което остава над топчето, докато то пада
    \item Грешна формула за сила на съпротивление в тръба 
    \item Топчето не е пуснато централно в тръбата
\end{itemize}
Балончето въздух може да се избегне като се пуска топчето от минимална височина. За да се установи вярността на поправката (умножаване на силата на съпротивление по $(1+2.4\frac{r}{R})$) трябва да се изследват тръби с различни радиуси и различни топчета.
\end{document}