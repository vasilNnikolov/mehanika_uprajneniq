\documentclass[aps, prb, twocolumn, a4paper]{revtex4-2}
\usepackage[T1,T2A]{fontenc}
\usepackage[utf8]{inputenc}
\usepackage[main=bulgarian, english]{babel}
\usepackage{lipsum}
\usepackage{gensymb}
\AtBeginDocument{\selectlanguage{bulgarian}}

\usepackage{amsmath}


\begin{document}
\title{Централен еластичен удар}
\author{Васил Николов}
\date{18.10.2021}
\maketitle

\section{Цел на експеримента}
Да се измери коефициентът на възстановяване на система от две метални топчета, окачени като махала и да се провери законът за запазване на импулса.

\section{Експериментална установка}
Две метални сфери с маси $m_1 = 180.5 g$ и $m_2=103.7 g$ са окачени на нишки с равна дължина така, че когато нишките са вертикални сферите леко да се опират. С електромагнит се пуска по-голямата от топките от определен ъгъл $\phi_0 = 13$\textdegree, и се измерват на око ъглите на максимално отклонение на двете топчета.

\section{Теоретична обосновка}
Коефициентът на възстановяване за тази система се дефинира с начанлата скорост на топката $v_1$ и крайните скорости на топчетата $u_1$ и $u_2$.
\begin{gather*}
    K=\frac{|u_2 - u_1|}{|v_1|}
\end{gather*}
При измерване на максималният ъгъл на отклонение на топчетата след удара, съответо $\phi_1$ и $\phi_2$, от ЗЗЕ следва, че 
\begin{gather*}
    u_1 = 2\sqrt{lg}sin(\frac{\phi_1}{2}) \\
    u_2 = 2\sqrt{lg}sin(\frac{\phi_2}{2}) \\
    K = \frac{sin(\frac{\phi_2}{2}) + sin(\frac{\phi_1}{2})}{sin(\frac{\phi_0}{2})} \tag{1}
\end{gather*}

За доказване на ЗЗИ трябва да проверим дали 
\begin{gather*}
    m_1sin(\frac{\phi_0}{2}) = m_1sin(\frac{\phi_1}{2}) + m_2sin(\frac{\phi_2}{2}) \tag{2}
\end{gather*}

За да проверим това трябва да сметнем стойностите на лявата и дясната страна на равенството, както и експерименталната грешка, която се смята по формулата
\begin{gather*}
    \Delta(msin(\frac{\phi}{2})) = sin(\frac{\phi}{2})\Delta m + \frac{1}{2}mcos(\frac{\phi}{2})\Delta \phi
\end{gather*}

\section{Екперименталлни данни и резултати}
Повтаря се измерването 20 пъти, като първите 10 се мери единият ъгъл на отклонение, а останалите - другият ъгъл.
\begin{table}[ht]
\begin{tabular}{llll}
No & $\phi_1$, deg & $\phi_2$, deg &  \\
1  & 4.5                    & 12.75                  &  \\
2  & 4.75                   & 13                     &  \\
3  & 4.5                    & 12.25                  &  \\
4  & 4.25                   & 12.75                  &  \\
5  & 4.25                   & 13.25                  &  \\
6  & 4.25                   & 13                     &  \\
7  & 4.25                   & 13.5                   &  \\
8  & 4.25                   & 13.25                  &  \\
9  & 4.25                   & 13.5                   &  \\
10 & 4                      & 13.75                  & 
\end{tabular}
\end{table}
От таблицата смятаме средните стойности на ъглите на отклонение
\begin{gather*}
    \bar{\phi_1} = 4.2 \degree, \bar{\phi_2} = 13.1 \degree, \Delta \phi \approx 0.25 \degree
\end{gather*}

По формулата (1) 
\begin{gather*}
    K = 0.67 \pm 8 \%
\end{gather*}

За проверка на законът за запазване на импулса трябва да се сметне лявата и дясната част на равенството, съответно $LHS$ и $RHS$.
\begin{gather*}
    LHS = (20.4 \pm 0.84) g \\
    RHS = (18.6 \pm 1.34) g
\end{gather*}

Двете рамки на неточността се припокриват, тоест експерименталните данни не откриват несъответствие със ЗЗЕ.

\section{Резултати}

Измерен е коефициентът на възстановяване на системата $K = 0.67 \pm 8\%$, и експерименталните данни са в съгласие с закона за запазване на импулса. 

\end{document}