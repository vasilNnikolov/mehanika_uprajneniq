\documentclass[%
 reprint,
 amsmath,amssymb,
 aps,
]{revtex4-2}
\usepackage{balance}
\usepackage[%
    margin=10mm,% ако не си принтира 10мм не изглежда грозно, а може да събереш повече текст
    % showframe=true,%
    ]{geometry}
\usepackage[T1,T2A]{fontenc}
\usepackage[utf8]{inputenc}
\usepackage[main=bulgarian, english]{babel}
\usepackage{float}
\AtBeginDocument{\selectlanguage{bulgarian}}
\newcommand{\degree}{^{\circ}}
\usepackage{amsmath}
\usepackage{graphics}
\usepackage{graphicx}
\graphicspath{{.}}
\newcommand{\abs}[1]{\lvert#1\rvert}
\let\phi\varphi
\usepackage{booktabs} % от тук се използва само \midrule може и без него 
\usepackage{dcolumn}
\newcolumntype{d}[1]{D{.}{.}{#1}}
\usepackage[unicode=true,pdfusetitle]{hyperref}
\usepackage[]{siunitx}

\usepackage[compact]{titlesec}

\begin{document}
\setlength{\abovedisplayskip}{3pt}
\setlength{\belowdisplayskip}{3pt}    

\title{Измерване на коефициентът на топлинно разширение на етилов спирт}
\author{Васил Николов}
\date{09.06.2022}
\maketitle
%%\balance
\section{Експериментална установка}

Установката се състои от голяма и малка колба, капилярка с диаметър $d = (1.5 \pm 0.08) \si{mm}$ и термометър. Голямата колба е пълна с вода, и лужи за водна баня на малката колба, в която е сложен спирт. Така се гарантира еднаква температура на спирта в целия обем на малката колба. Когато спиртът се нагрява обемът му се увеличава. Тъй като той се разширява повече от малката колба, спиртът влиза в капилярката, където можем да измерим височината му като функция на температурата.

\section{Теоретична обосновка}

При увели

\subsection{Адиабатно свиване}

В първата част бързо помпаме въздух в балона, и затваряме клапата между него и помпата. Тъй като процесът е бърз, то свиването е адиабатно. Нека в края налягането и температурата на газа в балона са съответно $p_1$ и $T_1$. Тогава можем да запишем

\begin{equation*}
    p_1^{1-\gamma} T_1^{\gamma} = p_0^{1 - \gamma} T_0^{\gamma}
\end{equation*}

Тук $p_0$ и $T_0$ са моментното атмосферно налягане и стайната температура.

\subsection{Квази-изохорно охлаждане}
При свиването на газа той се загрява, и ако оставим системата за няколко минути нейната температура ще се изравни с тази на околната среда. Обемът на системата се променя с това колко се променя височината на спиртния стълб в манометъра. Тъй като обаче този обем е на порядъци по-малък от обема на балона можем да приемем, че процесът е изохорен. Тогава можем да запишем

\begin{equation*}
    \frac{p_1}{T_1} = \frac{p_2}{T_0}
\end{equation*}

Когато се установи равновесие мерим налягането $p_2$ чрез манометъра. Нека то отговаря на височина $H_1$. 

\subsection{Адиабатно разширение}
Тук отваряме клапата, свързваща балонът с околната среда, за кратко време, под една секунда, и после затваряме клапата. Така газът в балона достига до налягане $p_0$ и температура $T_3$. Можем да запишем

\begin{equation*}
    p_2^{1-\gamma} T_0^{\gamma} = p_0^{1 - \gamma} T_3^{\gamma} \label{eq:1} \tag{1}
\end{equation*}

\subsection{Квази-изохорно затопляне}
Аналогично на част 2 процесът може да се приеме за изохорен. Нека изчакаме газът в балона да уравновеси температурата си с тази на околната среда, и нека измерим налягането му $p_4$. Вярно е, че

\begin{equation*}
    \frac{p_0}{T_3} = \frac{p_4}{T_0} \label{eq:2} \tag{2}
\end{equation*}

Нека на $p_4$ отговаря височина $H_2$. 

Обединявайки уравнения \eqref{eq:1} и \eqref{eq:2}, и представяйки $p_2$ и $p_4$ чрез $H_1$ и $H_2$ получаваме

\begin{gather*}
    (\frac{p_2}{p_0})^{1 - \gamma} = (\frac{T_3}{T_0})^{\gamma} = (\frac{p_0}{p_4})^{\gamma} \label{eq:3} \tag{3}\\
    p_2 = p_0 + \rho g H_1 \\
    p_4 = p_0 + \rho g H_2
\end{gather*}

$p_2$ и $p_4$ се различават много малко от $p_0$. Затова можем да използваме приближението на Бернули в уравнение \eqref{eq:3} и да го сведем до 

\begin{gather*}
    1 + (1 - \gamma) \frac{\rho g H_1}{p_0} = 1 - \gamma \frac{\rho g H_2}{p_0} \\
    (1 - \gamma) H_1 = -\gamma H_2 \\
    \gamma = \frac{H_1}{H_1 - H_2} \label{eq:4} \tag{4}
\end{gather*}

По уравнение \eqref{eq:4} ще пресметнем коефициентът на Поасон за въздух. 
\section{Експериментални данни и резултати}

Чрез повтаряне на гореописаният експеримент пет пъти получаваме стойност $\gamma = 1.34 \pm 0.2$, с гаранция $95 \%$. Това съвпата с очакванията ни $\gamma \approx 1.4$.

\end{document}

