\documentclass[%
 reprint,
 amsmath,amssymb,
 aps,
]{revtex4-2}
\usepackage{balance}
\usepackage[%
    margin=10mm,% ако не си принтира 10мм не изглежда грозно, а може да събереш повече текст
    % showframe=true,%
    ]{geometry}
\usepackage[T1,T2A]{fontenc}
\usepackage[utf8]{inputenc}
\usepackage[main=bulgarian, english]{babel}
\usepackage{float}
\AtBeginDocument{\selectlanguage{bulgarian}}
\newcommand{\degree}{^{\circ}}
\usepackage{amsmath}
\usepackage{graphics}
\usepackage{graphicx}
\graphicspath{{.}}
\newcommand{\abs}[1]{\lvert#1\rvert}
\let\phi\varphi
\usepackage{booktabs} % от тук се използва само \midrule може и без него 
\usepackage{dcolumn}
\newcolumntype{d}[1]{D{.}{.}{#1}}
\usepackage[unicode=true,pdfusetitle]{hyperref}
\usepackage[]{siunitx}

\usepackage[compact]{titlesec}


\begin{document}
\setlength{\abovedisplayskip}{3pt}
\setlength{\belowdisplayskip}{3pt}    

\title{Определяне на специфична топлина на изпарение на течен азот}
\author{Васил Николов}
\date{03.06.2022}
\maketitle
%%\balance

\section{Теоретична обосновка}

Целта на упражнението е експериментално да се определи специфичната топлина на изпарение на течен азот $\lambda$ при атмосферно налягане $P_0$.
При фазов преход на течен азот от течно към газообразно състояние той поглъща $\lambda = \frac{dQ}{dm}$ топлина на единица маса. $\lambda$ зависи от налягането, но в нашия случай то ще е атмосферно. 

\section{Експериментална установка}

Установката се състои от термоизолиран Дюаров съд, в който е потопен реотан. Този съд ще бъде пълен с течен азот. Към него е свързан воден манометър, който измерва налягането на системата, и цилиндър, които измерва обемът на газа, които се отделя от изпарението на течния азот. Експериментът се състои от две части. 

\subsection{Измерване на ефекта на околната среда}

В част А съдът с течният азот достига до термално равновесие, и се измерва колко газ се отделя за единица време. Този газ е заради неидеалната термална изолация на дюаровия съд. За масата изпарен газ за единица време можем да запишем следното равенство:

\begin{equation*} \label{eq:1}
    P_0 t = \lambda m_0 = \lambda \frac{P_{atm} \mu V_0}{RT} \tag{1}
\end{equation*}

Тук $P_0$ е енергията, която влиза в системата за единица време от околната среда, а $V_0$ е обемът на парите, които се изпаряват за $t=30 \ \si{s}$. 

\subsection{Измерване на количество изпарен азот с включен реотан}
Във втората част пускаме ток през реотана, и мерейки него и напрежението можем да сметнем отделената мощност в реотана, $P = IU$. Тъй като фазовият преход става при постоянна температура имаме същият ефект на топлина, влизаща от околната среда, както и в част A. Нека за $t=30 \ \si{s}$ се отделя обем $V$. Тогава 

\begin{gather*}
    (P + P_0) t = \lambda m = \lambda \frac{P_{atm}\mu V}{RT} \\
    P t = \lambda (m - m_0) = \lambda \frac{P_{atm}\mu (V - V_0)}{RT} \\
    \lambda = \frac{PtRT}{P_{atm}\mu(V - V_0)} \label{eq:2} \tag{2}
\end{gather*}

Използвайки уравнение \eqref{eq:2} можем да пресметнем специфичната топлина на фазовия преход по измерени обеми $V$ и $V_0$, както и останалите константни параметри на системата.
\section{Експериментални данни и резултати}



\end{document}

