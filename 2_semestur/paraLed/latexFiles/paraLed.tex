\documentclass[
 reprint,
 amsmath,amssymb,
 aps,
]{revtex4-2}
% \usepackage{balance}
\usepackage[%
    margin=10mm,% ако не си принтира 10мм не изглежда грозно, а може да събереш повече текст
    % showframe=true,%
    ]{geometry}
\usepackage[T1,T2A]{fontenc}
\usepackage[utf8]{inputenc}
\usepackage[main=bulgarian, english]{babel}
\usepackage{float}
\AtBeginDocument{\selectlanguage{bulgarian}}
\newcommand{\degree}{^{\circ}}
\usepackage{amsmath}
\usepackage{graphics}
\usepackage{graphicx}
\graphicspath{{.}}
\newcommand{\abs}[1]{\lvert#1\rvert}
\let\phi\varphi
\usepackage{booktabs} % от тук се използва само \midrule може и без него 
\usepackage{dcolumn}
\newcolumntype{d}[1]{D{.}{.}{#1}}
\usepackage[unicode=true,pdfusetitle]{hyperref}


\begin{document}
\setlength{\abovedisplayskip}{10pt}
\setlength{\belowdisplayskip}{10pt}    

\title{Експериментално установяване на специфичната топлина на топене на леда и специфичната топлина на изпарение на водна пара}
\author{Васил Николов}
\date{08.03.2022}
\maketitle

\section{Цел на упражнението}
Да се измерят специфичната топлина на топене на воден лед $\lambda$ и специфичната топлина на изпарение на водната пара $s$. 

\section{Теоретична обосновка}

При фазоф преход на водата от твърдо към течно състояние температурата е константна $(T = T_{melt})$, но за да разтопим лед с маса $m$ е нужно да се вкара допълнително енергия в системата $\Delta Q = \lambda m$. Тук $\lambda$ е константа - специфичната топлина на топене. Аналогично при изпарение на вода с маса $m$ можем да дефинираме $s = \frac{\Delta Q}{m} = const$. Тогава температурата на фазовият преход зависи от атмосферното налягане, което ще бъде измерено отделно. 

\section{Експериментална установка и работни формули}

\subsection{Специфична топлина на топене на леда}
В добре термоизолиран дюаров съд поставяме вода с начална температура $T_0 = 56.7 \pm 0.1 ^{\circ}C $ и маса $m_{H20} = 150.0 \pm 0.1 g$. Поставяме и няколко ледени кубчета и чакаме, докато се разтопят. Чрез потопеният в съда термометър измерваме температурата $T_1 = 26.1 \pm 0.1 ^{\circ}C $. След това премерваме масата на съда. Тъй като знаем масата му, когато е празен, и масата на водата в началото можем да определим точно масата на сложените ледени кубчета $m_{ice} = 49.0 \pm 0.1 g$. Установено е, че топлинният капацитет на дюаровият съд е равен на този на $m_d = 20g$ вода. Нека топлинният капацитет на водата е $C$. Тогава
\begin{gather*}
    c(m_{h2o} + m_d)(T_0 - T_1) + \lambda m_{ice} + c m_{ice} (T_1 - T_{melt}) = 0 \\
    \lambda = c \frac{(m_{h2o} + m_d)(T_0 - T_1) + m_{ice}(T_{melt} - T_1)}{m_{ice}}
\end{gather*}
С тези стойности пресмятаме крайната стойност на $\lambda = (331 \pm 4) \ kJ/kg$. Тази стойност е в съгласие с табличната стойност от $\lambda_0 = 334 \ kJ/kg$

\subsection{Специфична топлина на изпарение на водата}
В дюаровият съд сипваме дестилирана вода с начална температура $T_0 = 19.6 \pm 0.1^{\circ}C$. Потапяме маркуч на парогенераторът във водата, и го оставяме да работи докато температурата на водата не стигне $T_1 = 65.6^{\circ}C$. Изваждаме маркуча от водата и претегляме сместа, за да намерим масата на парата, която е кондензирала във водата. Началната маса на водата е $m_{h2o}=150.8g$, а масата на парата - $m_s = 14.2g$. 
\begin{gather*}
    c(m_{h2o} + m_d)(T_1 - T_0) - s m_s + c m_s (T_1 - T_{ph}) = 0 \\
    s = c \frac{(m_{h2o} + m_d)(T_1 - T_0) + m_s (T_1 - T_{ph})}{m_s}
\end{gather*}
Въздушното налягане по време на експеримента е $p = 710 mmHg$. При тази стойност температурата на изпарение на водата е $T_{ph} = 98.07^{\circ}C$. Тогава стойността на специфичната температура на изпарение на водата е $s = 2160 \pm 40 \ kJ/kg$. Тази стойност се различава с 5\% от истинската стойност за това налягане $s_0 = 2260 \ kJ/kg$.

\end{document}