\documentclass[reprint,amsmath,amssymb,aps,floatfix]{revtex4-2}
\usepackage{balance}
\usepackage[%
    margin=10mm,% ако не си принтира 10мм не изглежда грозно, а може да събереш повече текст
    % showframe=true,%
    ]{geometry}
\usepackage[T1,T2A]{fontenc}
\usepackage[utf8]{inputenc}
\usepackage[main=bulgarian, english]{babel}
\usepackage{float}
\AtBeginDocument{\selectlanguage{bulgarian}}
\newcommand{\degree}{^{\circ}}
\usepackage{amsmath}
\usepackage{graphics}
\usepackage{graphicx}
\graphicspath{{.}}
\usepackage{booktabs} % от тук се използва само \midrule може и без него 
\usepackage{dcolumn}
\usepackage{lipsum}
\newcolumntype{d}[1]{D{.}{.}{#1}}
\usepackage[unicode=true,pdfusetitle]{hyperref}
\usepackage[]{siunitx}
% \usepackage{balance}


% \usepackage[compact]{titlesec}

\begin{document}

\title{Повърхностно напрежение на смеси от спирт и вода}

\author{Васил Николов}
\date{26.04.2022}
\maketitle


\section{Цел на упражнението}

Да се определи по абсолютен и относителен метод коефициентът на повърхностно напрежение на смеси от вода и етилов спирт. 

\section{Експериментална установка}

Експериментална установка се състои от капилярка, чийто долен ръб докосва изследваната течност, а отгове е отворена към атвосферата. Течността е в затворен съд, като към капачката му е свързана и тръба, другият край на която е свързана към манометър, и налягането може да се изменя ????? . В експеримента се мери налягането, при което през капиляркакта започват да се отделят балончета. 

\section{Теоретична обосновка}
\subsection{Абсолютен метод}


\subsection{Относителен метод}


\section{Експериментални данни и резултати}
\subsection{Достигане на стационарно състояние}

За конкретната установка стационарното състояние се достига за около $15 \ \si{min}$. От графиката на Фигура \ref{fig:1} температурите на нагревателя и охладителя можем да видим крайните им стойности, $T_2 = (55.0 \pm 0.1) \degree C$ и $T_1 = (44.1 \pm 0.1) \degree C$. 


\subsection{Измерване на скоростта на охлаждане на охладителя}

На Фигура \ref{fig:2} е представена температурата на охладителя като функция на времето. За да намерим нейната числена производна в точката, където температурата е числено равна на $T_2 = 40.4 \ \degree C$ фитираме полином от пета степен на експерименталните данни, и намираме аналитично неговата производна.  


Използвайки формула \eqref{eq:4} пресмятаме крайната стойност за коефициенът на топлопроводност на образеца - $k = 0.11 \ \si{Wm^{-1}K^{-1}} \pm 3\%$.  За да се пресметне грешката се предполага, че грешката в производната на температурата е около $1\%$. Този резултат е очакван - топлопроводимостта на образеца е от същия порядък като тази на плексиглас. 

\end{document}

