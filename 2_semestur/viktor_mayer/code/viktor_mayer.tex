\documentclass[%
 reprint,
 amsmath,amssymb,
 aps,
]{revtex4-2}
\usepackage{balance}
\usepackage[%
    margin=10mm,% ако не си принтира 10мм не изглежда грозно, а може да събереш повече текст
    % showframe=true,%
    ]{geometry}
\usepackage[T1,T2A]{fontenc}
\usepackage[utf8]{inputenc}
\usepackage[main=bulgarian, english]{babel}
\usepackage{float}
\AtBeginDocument{\selectlanguage{bulgarian}}
\newcommand{\degree}{^{\circ}}
\usepackage{amsmath}
\usepackage{graphics}
\usepackage{graphicx}
\graphicspath{{.}}
\newcommand{\abs}[1]{\lvert#1\rvert}
\let\phi\varphi
\usepackage{booktabs} % от тук се използва само \midrule може и без него 
\usepackage{dcolumn}
\newcolumntype{d}[1]{D{.}{.}{#1}}
\usepackage[unicode=true,pdfusetitle]{hyperref}
\usepackage[]{siunitx}

\usepackage[compact]{titlesec}


\begin{document}
\setlength{\abovedisplayskip}{3pt}
\setlength{\belowdisplayskip}{3pt}    

\title{Определяне на относителната плътност на парите на хлороформ по метода на Виктор Майер}
\author{Васил Николов}
\date{03.06.2022}
\maketitle
%%\balance
\section{Цел на упражнението}

Да се измери относителната плътност на парите на хлороформ по метода на Виктор Майер. Представяме резултатите в относителен вид $\theta = \frac{\rho}{\rho_0}$ за да получим безразмерна величина, която лесно можем да сравняваме, и е лесна за интуитивно разбиране. Тук $\rho_0$ е плътността на сух въздух при стандартни условия.

\section{Експериментална установка}

Установката се състои от дълга и тясна епруветка, в която ще сложим отворена ампула с хлороформ. Епруветката е запушена, и свързана с маркуч. Около тази епруветка има по-голяма и деебла стъклена тръба, в която се пуска пара, създадена от парогенератор. Голямата тръба действа като изолатор за малката, за да може тя да се загрее достатъчно. Маркучът е потопен във вана с вода. Във ваната се поставя вертикална епруветка, която е напълнена предварително с вода. По този начин когато я обърнем с дъното нагоре тя ще е пълна с вода. Свободният край на маркуча се пъха в отвора на потопената епруветка, така че газът, който излиза, ще остане в нея. 

\section{Теоретична обосновка}

Нека след като сложим ампулата с хлороформ и се установи равновесие обемът на газът в обърнатата епруветка е $V$, височината на водният стълб е $h$, а налягането на наситените водни пари за стайна температура е $e$. Знаем масата на хлороформът $M$. Нека налягането в обърнатата епруветка е $p$.

\begin{gather*}
    \frac{p_0 V_0}{T_0} = \frac{p V}{T} \\
    V_0 = \frac{p V T_0}{p_0 T} \\
    \theta = \frac{\rho}{\rho_0} = \frac{M}{V_0 \rho_0} = \frac{M p_0 T}{p V T_0 \rho_0}\\
    p + e + \rho_w g h = p_{atm} \\
    \theta = \frac{M p_0 T}{(p_{atm} - e - \rho g h) V T_0 \rho_0} \label{eq:1} \tag{1}
\end{gather*}

По уравнение \eqref{eq:1} ще пресметнем относителната плътност на парите на хлороформът, като знаем, че $p_0 = 10^5 \ Pa$, $T_0 = 0\degree C$, $\rho_0 = 1.293 \ \si{kg.m^{-3}}$.

\section{Експериментални данни и резултати}

От установката измерваме стайната температура $T = 26\degree \ C$, $M = 59.56 \ \si{mg}$, $p_{atm} = 713.1 \si{mmHg} = 95070 \ \si{Pa}$. При тази стайна температура налягането на наситените водни пари е $e = 3365 \ \si{Pa}$. Тогава по формула \eqref{eq:1} пресмятаме, че относителната маса на парите на хлороформът е $\theta = 5.66 \pm 3\%$.

\end{document}

