\documentclass[aps, prb, twocolumn, a4paper, floatfix, reprint]{revtex4-2}
\usepackage{balance}
\usepackage[%
    margin=10mm,% ако не си принтира 10мм не изглежда грозно, а може да събереш повече текст
    % showframe=true,%
    ]{geometry}
\usepackage[T1,T2A]{fontenc}
\usepackage[utf8]{inputenc}
\usepackage[main=bulgarian, english]{babel}
\usepackage{float}
\AtBeginDocument{\selectlanguage{bulgarian}}
\newcommand{\degree}{^{\circ}}
\usepackage{amsmath}
\usepackage{graphics}
\usepackage{graphicx}
\graphicspath{{.}}
\newcommand{\abs}[1]{\lvert#1\rvert}
\let\phi\varphi
\usepackage{booktabs} % от тук се използва само \midrule може и без него 
\usepackage{dcolumn}
\newcolumntype{d}[1]{D{.}{.}{#1}}
\usepackage[unicode=true,pdfusetitle]{hyperref}

\usepackage[compact]{titlesec}
\titlespacing{\section}{0pt}{2ex}{1ex}
\titlespacing{\subsection}{0pt}{1ex}{0ex}
\titlespacing{\subsubsection}{0pt}{0.5ex}{0ex}


\begin{document}

\title{Термодвойка}
\author{Васил Николов}
\date{08.03.2022}
\maketitle
%%\balance

\section{Цел на упражнението}
Да се наблюдава втвърдяването на смес на Вуд, и да се изследва и калибрира термодвойка, потопена в сместа. 

\section{Експериментална установка}
Едната част на диференциална термодвойка е потопена в термос с вода и лед, а другата - в колба съдържаща сплав на Вуд. Върху сплавта е изсипан глицерин, за да се избегне окисляване, и цялата колба е потопена във вода, която може да се нагрява от котлон. Термодвойката е свързана към усилвател на напрежението, който го превръща в сигнал от порядъка на десетки миливолти, който може да се измери с мултицет с точност $\delta U_0 = 0.1 \ mV$. Тъй като единият край на термодвойката е потопен в среда с температура $T_0 = 0^{\circ}C$, то напрежението от усилвателя отговаря на температурата на сместа в целзиеви градуси. След като колбата със сместа се нагрее до $T_{max} = 92^{\circ} C$ котлонът се премахва и сместа се охлажда. Правят се измервания докато се стигне до $T_{min} = 52^{\circ} C$, като в този температурен интервал е и температурата на фазофия преход на сместа на Вуд, около $T_{ph} = 69^{\circ} C$. За калибрация на термодвойката се използва живачен термометър, който също е потопен в сплавта. 

\section{Теоретична обосновка}
\subsection{Градуиране на термодвойката}
Заради неидеалност на усилвателя и ефекта на Зийбек не може да очакваме идеална зависимост на термоелектродвижещото напрежение от разликата в температурите. 
\begin{gather*}
    U_{ideal} = K\Delta T \\
    U_{real} = a\Delta T^2 + b\Delta T + c \\
\end{gather*}
Измереното термоелектродвижещото напрежение се приближава с полином от втора степен защото по-високи степени не добавят точност в модела. 

\subsection{Фазов преход на сплав на Вуд}
При охлаждането на сместа очакваме експоненциално намаляване на температурата до достигането на фазовия преход, плато на температурата при температурата на фазовия преход и отново експоненциално намаляване след него. Следователно очакваме функцията на температурата от времето да е следната
\begin{equation*} \label{eq:1}
    T(t) = \begin{cases}
        T_r + (T_0 - T_r) e^{-\lambda t} \quad \text{for } t < t_1 \\
        T_{ph} \quad \text{for } t_1 < t < t_2 \\
        T_r + (T_{ph} - T_r) e^{-\lambda (t - t_2)} \quad \text{for } t > t_2
    \end{cases} \tag{1}
\end{equation*}
Тук $T_r$ е стайната температура. Фитираме такава крива по метода на най-малките квадрати, за да намерим оптималните стойности на параметрите $\lambda, t_1, t_2 \text{ и } T_r$. Трябва да отчетем, че
\begin{equation*}
    T_{ph} = T_r + (T_0 - T_r) e^{-\lambda t_1}
\end{equation*}
Така можем да определим точно кога започва и приключва фазовият преход, и при каква температура се случва той. 

\section{Експериментални данни и резултати}
\subsection{Градуиране на термодвойката}
На графиката са представени измерените стойности за термоелектродвижещото напрежение като функция на температурата, както и най-добре описващата ги парабола. 

\begin{figure}[H]
    \centering
    \caption{Термоелектродвижещо напрежение като функция на температура}
    \includegraphics[width=0.9\columnwidth, keepaspectratio=true]{graduirane.png}
\end{figure}

Параметрите на калибровъчната крива са 
\begin{equation*}
    a = -1.296.10^{-3} \frac{mV}{K^2}, \ b = 1.147 \frac{mV}{K}, \ c = -29.89 mV
\end{equation*}
Съобразявайки в коя част на парабобата е интервалът на измервания можем да пресметнем офратната функция 
\begin{equation*} \label{eq:2}
    T = T(E) = \frac{-b+\sqrt{b^2 - 4a(c-E)}}{2a} \tag{2}
\end{equation*}

Грешката при определяне на температурата по този начин е 
\begin{equation*}
    \delta T = \sqrt{T_{rmse}^2 + (\frac{\delta U_0}{b})^2} = 0.3K
\end{equation*}
Тук $T_{rmse}$ е средната квадратична грешка на пресметнатата температура. 

\subsection{Фазов преход на сплав на Вуд}
На фигура 2 е представена графика на температурата на сплавта като функция на времето, и най-добрата функция от вид \eqref{eq:1}
 
\begin{figure}[H]
    \centering
    \caption{Температура на сплавта като функция на времето}
    \includegraphics[width=0.95\columnwidth, keepaspectratio=true]{cool3.png}
\end{figure}
Платото на температурата се получава при $T_{ph} = 67.9^{\circ}C$. Това е температурата на фазовия преход за тази смес на Вуд. Средата на платото е около $t_{mid}=1600s$. В този момент напрежението е $E_{mid} = 42.4 mV$. Това отговаря на температура $T_{th} = 68.3^{\circ} C$, пресметнато по \eqref{eq:2}. Ако термодвойката е калибрирана правилно то 
\begin{gather*}
    \vert T_{th} - T_{mid} \vert < \delta T_{term} + \delta T \\
    \vert T_{th} - T_{mid} \vert = 0.4^{\circ}C \\
    \delta T_{term} + \delta T = 0.8^{\circ}C 
\end{gather*}
Следователно калибрацията е точна. 
\end{document}